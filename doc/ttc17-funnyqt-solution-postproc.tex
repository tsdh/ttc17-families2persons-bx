\documentclass[a4paper]{article}
\usepackage{onecolpceurws}

\usepackage[T1]{fontenc}
\usepackage{varioref}
\usepackage{hyperref}

\usepackage{url}
\usepackage{paralist}
\usepackage{graphicx}
\usepackage{todonotes}

\usepackage{relsize}
\usepackage[cache]{minted}
\setminted{fontsize=\fontsize{8}{8},breaklines=true,breakbytoken=true,numbersep=5pt,numberblanklines=false}
\setmintedinline{fontsize=\relscale{.9},breaklines=true,breakbytoken=true}
\newminted{clojure}{}
\newmintinline{clojure}{}
\newminted{java}{}
\newcommand{\code}{\clojureinline}
\VerbatimFootnotes

\usepackage{bigfoot}
\DeclareNewFootnote[para]{default}

\title{Solving the TTC Families to Persons Case with FunnyQT}
\author{Tassilo Horn\\ \href{mailto:tsdh@gnu.org}{tsdh@gnu.org}}
\institution{The GNU Project}

\pagestyle{empty}

\begin{document}
\maketitle

\begin{abstract}
  This paper describes the FunnyQT solution to the bidirectional Families to
  Persons transformation case of TTC 2017.  The solution is simple and concise
  and passes all batch transformation and some incremental tests.
\end{abstract}


\section{Introduction}
\label{sec:introduction}

This paper describes the FunnyQT\footnote{\url{http://funnyqt.org}}
~\cite{diss,funnyqt-icgt15} solution of the Families to Persons
case~\cite{f2p-case-desc} of TTC 2017.  With only 52 lines of declarative code,
the solution is able to pass all batch transformation tests and some
incremental tests.  The solution project is available on
Github\footnote{\url{https://github.com/tsdh/ttc17-families2persons-bx}} and it
is integrated into the
\emph{benchmarx}\footnote{\url{https://github.com/eMoflon/benchmarx/}}
framework.

FunnyQT is a model querying and transformation library for the functional Lisp
dialect Clojure\footnote{\url{http://clojure.org}}.  Queries and
transformations are Clojure functions using the features provided by the
FunnyQT API.

Clojure provides strong metaprogramming capabilities that are used by FunnyQT
in order to define several \emph{embedded domain-specific languages} (embedded
DSL~\cite{book:Fowler2010DSL}) for different querying and transformation tasks.

FunnyQT is designed to be extensible.  By default, it supports
EMF~\cite{Steinberg2008EEM} and
JGraLab\footnote{\url{https://github.com/jgralab/jgralab}} TGraph models.
Support for other modeling frameworks can be added without having to touch
FunnyQT's internals.

The FunnyQT API is structured into several namespaces, each namespace providing
constructs supporting concrete querying and transformation use-cases, e.g.,
model management, functional querying, polymorphic functions, relational
querying, pattern matching, in-place transformations, out-place
transformations, bidirectional transformations, and some more.  For solving the
Families to Persons case, its bidirectional transformation and relational model
querying DSLs have been used.


\section{Solution Description}
\label{sec:solution-description}

This section explains the FunnyQT solution.  First,
Section~\ref{sec:f2p-solution} explains the actual transformation solving the
case.  Then, Section~\ref{sec:gluing} explains how the solution is integrated
into the \emph{benchmarx} framework.


\subsection{The Families to Persons Transformation}
\label{sec:f2p-solution}

As a first step, relational querying APIs for the two metamodels are generated.

\begin{clojurecode*}{linenos=true}
(rel/generate-metamodel-relations "metamodels/Families.ecore" f)
(rel/generate-metamodel-relations "metamodels/Persons.ecore" p)
\end{clojurecode*}

This makes the relations for the families metamodel available with the
namespace alias \code|f| and those of the persons metamodel with alias
\code|p|.  Relations are generated for every metamodel element.  For example,
there is a relation \code|(f/Family f family)| which succeeds for every
\code|family| in the given family model \code|f|.  There is a relation
\code|(f/name f el n)| which succeeds if \code|el| is a member or family of the
family model \code|f| whose name is \code|n|.  And there are reference
relations like \code|(f/->sons f parent child)| which succeed if \code|child|
is a son of \code|parent|.

Before the transformation itself, we define a helper relation which defines the
possible kinds of relationships between a \code|family| and a family
\code|member| depending on if we prefer to create parents over creating
children (parameter \code|pref-parent|).  This is a higher-order relation in
that the two remaining parameters are a parent relation \code|prel| (either
\code|f/->father| or \code|f/->mother|) and a child relation (either
\code|f/->daughters| or \code|f/->sons|).

\begin{clojurecode*}{linenos=true,firstnumber=last}
(defn relationshipo [pref-parent f family member prel crel]
  (ccl/conda
   [(bx/target-directiono :right)     ;; (1)
    (ccl/conde
     [(prel f family member)]
     [(crel f family member)])]
   [(bx/existing-elemento? member)]   ;; (2)
   [(ccl/== pref-parent false)        ;; (3)
    (crel f family member)]
   [(bx/unseto? f family prel member) ;; (4)
    (prel f family member)]
   [(crel f family member)]))         ;; (5)
\end{clojurecode*}

\code|ccl/conda| is like a short-cutting logical disjunction.  The \(n\)-th
clause is only tried if all preceeding clauses fail.  The first clause succeeds
when we are transforming into the direction of the right model, i.e., the
person register.  In this case, \code|member| may be in a parental role of
family (\code|prel|), or it might be in a child role (\code|crel|).  We don't
really care but want to ensure that all members of the given \code|family| are
reachable, thus we use a non-short-cutting \code|ccl/conde| which gives both
clauses a chance to succeed.

All other clauses deal with transforming a person model to a family model,
i.e., the backward transformation.  The second clause deals with the case where
\code|member| is an already existing element, i.e., not created by the current
execution of the transformation.  Here, we assume that this member is already
properly assigned to a family, so we simply succeed without doing anything.

In the third clause, if we do not prefer assigning to parental roles, then the
child relation \code|crel| must succeed between the \code|family| and the
\code|member|.  The child relations can always succeed because a family can
have an arbitrary number of daughters and sons.  Thus, in the remaining clauses
we have to deal only with the case where assigning to parental roles is
preferred.

In the fourth clause, if the \code|family|'s parental role is still unset or
already assigned to \code|member|, then the parental relation must succeed
between the \code|family| and the \code|member|.

Lastly, if no clause has succeeded until now, then the child relation has to
succeed.  As said, this goal\footnote{The application of a relation is called a
  \emph{goal}.} can always succeed.

\bigskip{}

In the following, the actual transformation definition is explained.  It is
defined using the \code|deftransformation| macro provided by FunnyQT.

\begin{clojurecode*}{linenos=true,firstnumber=last}
(bx/deftransformation families2persons [f p prefer-parent prefer-ex-family]
  :delete-unmatched-target-elements true
  :id-init-fn bx/number-all-source-model-elements
\end{clojurecode*}

The name of the transformation is \code|families2persons| and it declares four
parameters.  The parameter \code|f| is the family model (the left model),
\code|p| is the persons model (the right model), \code|prefer-parent| is a
Boolean flag determining if we prefer creating parents to creating children,
and \code|prefer-ex-family| is a Boolean flag, too, determining if we prefer
re-using existing families over creating new families for new family members.

By default, bidirectional FunnyQT transformations will never delete elements
from the current target model, i.e., the model being created or modified by the
synchronization.  The reason for that behavior is that it allows to run the
transformation in one direction first and then in the other direction in order
to perform a full synchronization where missing elements are created in each of
the two models.  Thus, after running a transformation, e.g., in the direction
of the right model, it is only ensured that for each element (considered by the
transformation's rules) in the left model, there is a corresponding counterpart
in the right model.  However, the right model might still contain elements
which have no counterpart in the left model.  With option
\code|:delete-unmatched-target-elements| set to \code|true|, this behavior is
changed.  Elements in the current target model which are not required by the
current source model and the transformation relations are deleted.

The next option, \code|:id-init-fn|, has the following purpose.  In this
transformation case, family members and persons do not have some kind of unique
identity.  For example, it is allowed to have two members named Jim with the
same name in the very same family Smith.  However, FunnyQT BX transformations
have forall-there-exists semantics where it would suffice to create just one
person in the right model with the name set to ``Smith, Jim''.  But the case
description mandates that we create one person for every member and vice versa,
no matter if they can be distinguished based on their attribute values.  For
such scenarios, FunnyQT's bidirectional transformation DSL provides a concept
of synthetic ID attributes.  The value of \code|:id-init-fn| has to be a
function which returns a map from elements to their synthetic IDs.  The
built-in function \code|bx/number-all-source-model-elements| returns a map
where every element in the source model gets assigned a unique integer number.
These synthetic IDs are then used in a transformation relation which is
discussed further below.

The first transformation relation, \code|family-register2person-register|,
transforms between family and person registers.

\begin{clojurecode*}{linenos=true,firstnumber=last}
  (^:top family-register2person-register
   :left  [(f/FamilyRegister f ?family-register)]
   :right [(p/PersonRegister p ?person-register)]
   :where [(member2female :?family-register ?family-register :?person-register ?person-register)
           (member2male :?family-register ?family-register :?person-register ?person-register)])
\end{clojurecode*}

It is defined as a top-level rule meaning that it will be executed as the entry
point of the transformation.  Its \code|:left| and \code|:right| clauses
describe that for every \code|?family-register| there has to be a
\code|?person-register| and vice versa.  We assume that there is always just
one register in each model.

The \code|:where| clause defines that after this relation has been enforced (or
checked in checkonly mode), then the two transformation relations
\code|member2female| and \code|member2male| have to be enforced (or tested)
between the current \code|?family-register| and
\code|?person-register|\footnote{Transformation relations are called with
  keyword parameters.  The two calls in the \code|:where| clause state that the
  current \code|?family-register| will be bound to the logic variable with the
  same name in the called relation and the same is true for the
  \code|?person-register|.}.

The next transformation relation, \code|member2person|, describes how family
members of a family contained in a family register in the left model correspond
to persons contained in a person register in the right model.  As it can be
seen, there is no goal describing how the \code|?family| and the \code|?member|
are connected in the \code|:left| clause and in the \code|:right| clause we are
dealing with just a \code|?person| of type \code|Person| which is an abstract
class.  As such, this relation is not sufficient for the complete
synchronization between members in the different roles of a family to females
and males.  Instead, it only captures the aspects that are common in the cases
where mothers and daughters are synchronized with females and fathers and sons
are synchronized with males.  Therefore, this relation is declared abstract.

\begin{clojurecode*}{linenos=true,firstnumber=last}
  (^:abstract member2person
   :left  [(f/->families f ?family-register ?family)
           (f/Family f ?family)
           (f/name f ?family ?last-name)
           (f/FamilyMember f ?member)
           (f/name f ?member ?first-name)
           (id ?member ?id)
           (ccl/conda
            [(ccl/== prefer-ex-family true)]
            [(bx/existing-elemento? ?member)
             (id ?family ?last-name)]
            [(id ?family ?id)])]
   :right [(p/->persons p ?person-register ?person)
           (p/Person p ?person)
           (p/name p ?person ?full-name)
           (id ?person ?id)]
   :when  [(rel/stro ?last-name ", " ?first-name ?full-name)])
\end{clojurecode*}

So what are these common aspects?  Well, a \code|?member| of a \code|?family|
(where we have not determined the role, yet) contained in the
\code|?family-register| passed in as parameter from
\code|family-register2person-register| corresponds to a \code|?person| (where
we have not determined the gender, yet) contained in the
\code|?person-register| passed in as the other parameter from
\code|family-register2person-register|.  The \code|:when| clause defines that
the concatenation of the \code|?family|'s \code|?last-name|, the string
\code|", "| and the \code|?member|'s \code|?first-name| gives the
\code|?full-name| of the \code|?person|.

What has not been described so far are the \code|id| goals in lines 29, and 34
and the \code|ccl/conda| goal starting in line 30.  The first two define that
the \code|?member| and the corresponding \code|?person| must have the same
synthetic ID.  Remember the \code|:id-init-fn| in line 17 which assigned a
unique number to every element in the respective source model of the
transformation.  With these synthetic IDs, the transformation is able to create
one person for every member and vice versa even in the case where two elements
are equal based on their attribute values.

Lastly, the \code|ccl/conda| goal starting in line 30 of the \code|:left|
clause handles the preference of re-using existing families, i.e., assigning
new members to existing families, over creating new families for new members.
By default, FunnyQT would always try to re-use an existing family.  Thus, if
the \code|prefer-ex-family| parameter is \code|true|, nothing needs to be done.
Likewise, if \code|?member| is an existing element for which we assume she is
already assigned to some family, we can also just stick to the default behavior
but define the ID of the \code|?family| to be its name (although it is probably
not unique).  If the first two \code|ccl/conda| clauses fail, i.e.,
\code|prefer-ex-family| is \code|false| and \code|?member| is a new member
which is just going to be created by the enforcement of this relation, then we
define that the ID of the \code|?family| must equal to the IDs of the
\code|?member| and \code|?person|.  Thus, in this case and only in this case,
new members force the creation of a new family even when there is already a
family with the right name.

The last two transformation relations extend the \code|member2person| relation
for synchronizing between members in the role of a family mother or daughter
and female persons and between family fathers or sons and male persons.

\begin{clojurecode*}{linenos=true,firstnumber=last}
  (member2female
   :extends [(member2person)]
   :left  [(relationshipo prefer-parent f ?family ?member f/->mother f/->daughters)]
   :right [(p/Female p ?person)])
  (member2male
   :extends [(member2person)]
   :left  [(relationshipo prefer-parent f ?family ?member f/->father f/->sons)]
   :right [(p/Male p ?person)]))
\end{clojurecode*}

In the \code|:left| clauses we use the \code|relationshipo| helper relation
described in the beginning of this section which chooses the right female or
male role based on the preference parameter \code|prefer-parent| and the
current state of the family, i.e., by checking if the respective parental role
is still unset.  In the two \code|:right| clauses, we only need to specify that
the \code|Person| \code|?person| is actually a \code|Female| or \code|Male|.

These 33 lines of transformation specification plus the 12 lines for the
\code|relationshipo| helper, and two lines for the generation of the
metamodel-specific relational querying APIs form the complete functional parts
of the solution.  The only thing omitted from the paper are the namespace
declarations\footnote{The Clojure equivalent of Java's package statement and
  imports.}  consisting of 5 lines of code.


\subsection{Gluing the Solution with the Framework}
\label{sec:gluing}

Typically, open-source Clojure libraries and programs are distributed as JAR
files that contain the source files rather than byte-compiled class files.
This solution does almost the same except that the JAR contains the solution
source code, FunnyQT itself (also as sources) and every dependency of FunnyQT
(like Clojure) except for EMF which the \emph{benchmarx} project already
provides.

Calling Clojure functions from Java is really easy and FunnyQT transformations
are no exception because they are plain Clojure functions, too.  The
\code|BXTool| implementation \code|FunnyQTFamiliesToPerson| extends the
\code|BXToolForEMF| class.  Essentially, it has just a static member \code|T|
which is set to the transformation.

\begin{javacode}
public class FunnyQTFamiliesToPerson extends BXToolForEMF<FamilyRegister, PersonRegister, Decisions> {
    private final static Keyword LEFT = (Keyword) Clojure.read(":left");
    private final static Keyword RIGHT = (Keyword) Clojure.read(":right");
    private final static IFn T; // <-- The FunnyQT Transformation

    static {
        final String transformationNamespace = "ttc17-families2persons-bx.core";
        // Clojure's require is similar to Java's import.  However, it also loads the required
        // namespace from a source code file and immediately compiles it.
        final IFn require = Clojure.var("clojure.core", "require");
        require.invoke(Clojure.read(transformationNamespace));
        T = Clojure.var(transformationNamespace, "families2persons");
    }
\end{javacode}

All Clojure functions implement the \code|IFn| interface and can be called
using \code|invoke()|.  And exactly this is done to call the transformation.

\begin{javacode}
    private void transform(Keyword direction) {
        T.invoke(srcModel, trgModel, direction,
                 configurator.decide(Decisions.PREFER_CREATING_PARENT_TO_CHILD),
                 configurator.decide(Decisions.PREFER_EXISTING_FAMILY_TO_NEW));
    }
\end{javacode}

This corresponds to a direct Clojure call
\code|(families2persons src trg dir prefer-parent prefer-ex-family)|.

\section{Evaluation and Conclusion}
\label{sec:evaluation}

In this section, the test results of the FunnyQT solution are presented and
classified as requested by the case description~\cite{f2p-case-desc}.  Since
the bidirectional transformation DSL of FunnyQT is state-based and not
incremental at all (and by design), many of the incremental tests are mostly
out of scope.  However, in its core use case, non-incremental bidirectional
transformations, the solution passes all tests.

\begin{compactdesc}
\item[BatchForward.*] All tests result in an \emph{expected pass}.
\item[BatchBwdEandP.*] All tests result in an \emph{expected pass}.
\item[BatchBwdEnotP.*] All tests result in an \emph{expected pass}.
\item[BatchBwdNotEandP.*] All tests result in an \emph{expected pass}.
\item[BatchBwdNotEnotP.*] All tests result in an \emph{expected pass}.
\item[IncrementalForward.*] The solution \emph{expectedly passes} the tests
  \code|testStability|, \code|testHippocraticness|, and
  \code|testIncrementalInserts|.  All remaining tests \emph{fail expectedly}
  because those tests require incremental abilities.  For example, in some
  tests the birthdates are set manually in the persons model.  The re-execution
  of the transformation causes deletion and re-creation of those persons,
  however then they get assigned default birthdates and not the manually edited
  ones as the transformation does not consider this attribute at all.
\item[IncrementalBackward.*] The solution \emph{expectedly passes} the tests
  \code|testStability|, \code|testIncrementalInsertsFixedConfig|,
  \code|testIncrementalOperational|, and \code|testHippocraticness|.  The test
  \code|testIncrementalInsertsDynamicConfig| \emph{fails unexpectedly}.  When
  we neither prefer existing families nor assigning to parental roles, the
  transformation still adds the new Seymore to an existing family in a parental
  role.  All other tests are \emph{expected fails} due to the fact that they
  require incremental capabilities.
\end{compactdesc}

In summary, the correctness is satisfying.  There is only one test which fails
unexpectedly.  All other fails are expected and can hardly be solved by a
non-incremental approach.

A very weak point of the solution is its performance.  It is at least an order
of magnitude slower than the other solutions already integrated in the
benchmarx project (BiGUL, eMoflon, BXtend, MediniQVT).  Where their runtimes
are in the tenth of seconds or below, the FunnyQT solution takes seconds.  With
models in the size of thousands of elements, you might have to wait a bit for
the transformation to finish.  One reason is that the FunnyQT BX implementation
is state-based and thus needs to recompute the correspondences between all
elements in the source and target models on every invocation instead of just
propagating small deltas like incremental approaches do.  Furthermore,
FunnyQT's bidirectional transformation DSL is built upon the relational
programming library
\emph{core.logic}\footnote{\url{https://github.com/clojure/core.logic}} which
is not tuned for performance but for simplicity and extensibility of its
implementation and FunnyQT probably does not use it in the best possible way.

Other good points of the solution are its conciseness and simplicity.  With
only 52 lines of code, it is by far the most concise solution currently
integrated in the benchmarx project.  And it is quite simple to understand.
The only complexities it has arose from the different alternatives depending on
external parameters.

\bibliographystyle{alpha}
\bibliography{ttc17-funnyqt-solution}

\end{document}

%%% Local Variables:
%%% mode: latex
%%% TeX-master: t
%%% TeX-command-extra-options: "-shell-escape"
%%% LaTeX-verbatim-macros-with-delims-local: ("code")
%%% End:

%  LocalWords:  parallelizes
