\documentclass[a4paper]{article}
\usepackage{onecolceurws}

\usepackage[T1]{fontenc}
\usepackage{varioref}
\usepackage{hyperref}

\usepackage{url}
\usepackage{paralist}
\usepackage{graphicx}
\usepackage{todonotes}

\usepackage{relsize}
\usepackage[cache]{minted}
\setminted{fontsize=\fontsize{8}{8},breaklines=true,breakbytoken=true,autogobble,linenos,numbersep=3pt,numberblanklines=false,firstnumber=last}
\setmintedinline{fontsize=\relscale{.9},breaklines=true,breakbytoken=true}
\newminted{clojure}{}
\newmintinline{clojure}{}
\newcommand{\code}{\clojureinline}
\VerbatimFootnotes

\title{Solving the TTC Families to Persons Case with FunnyQT}
\author{Tassilo Horn\\
  \href{tsdh@gnu.org}{mailto:tsdh@gnu.org}}

\begin{document}
%%\maketitle

\begin{abstract}
  This paper describes the FunnyQT solution to the TTC 2017 Families to Persons
  bidirectional transformation case.
\end{abstract}


\section{Introduction}
\label{sec:introduction}

This paper describes the FunnyQT\footnote{\url{http://funnyqt.org}}
~\cite{diss,funnyqt-icgt15} solution of the TTC 2017 Families to Persons
case~\cite{f2p-case-desc}.

\todo[inline]{What has been solved?}

The solution project is available on
Github\footnote{\url{https://github.com/tsdh/ttc17-families2persons-bx}}.

\todo[inline]{What about SHARE???}

FunnyQT is a model querying and transformation library for the functional Lisp
dialect Clojure\footnote{\url{http://clojure.org}}.  Queries and
transformations are Clojure programs using the features provided by the FunnyQT
API.

Clojure provides strong metaprogramming capabilities that are used by FunnyQT
in order to define several \emph{embedded domain-specific languages} (DSL) for
different querying and transformation tasks.

FunnyQT is designed with extensibility in mind.  By default, it supports EMF
models and JGraLab TGraph models.  Support for other modeling frameworks can be
added without having to touch FunnyQT's internals.

The FunnyQT API is structured into several namespaces, each namespace providing
constructs supporting concrete querying and transformation use-cases, e.g.,
model management, functional querying, polymorphic functions, relational
querying, pattern matching, in-place transformations, out-place
transformations, bidirectional transformations, and some more.  For solving the
families to persons case, its bidirectional transformation and relational model
querying DSLs have been used.


\section{Solution Description}
\label{sec:solution-description}



\subsection{Gluing the Solution with the Framework}
\label{sec:gluing}

Typically, open-source Clojure libraries and programs are distributed as JAR
files that contain the source files rather than byte-compiled class files.
This solution does the same, and that JAR is deployed to a local Maven
repository from which the Maven build infrastructure of the benchmark framework
can pick it up.



\section{Evaluation \& Conclusion}
\label{sec:evaluation}



\bibliographystyle{alpha}
\bibliography{ttc17-funnyqt-solution}

\end{document}

%%% Local Variables:
%%% mode: latex
%%% TeX-master: t
%%% TeX-command-extra-options: "-shell-escape"
%%% LaTeX-verbatim-macros-with-delims-local: ("code")
%%% End:

%  LocalWords:  parallelizes
